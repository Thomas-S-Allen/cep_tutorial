\documentclass[]{article}
\usepackage{lmodern}
\usepackage{amssymb,amsmath}
\usepackage{ifxetex,ifluatex}
\usepackage{fixltx2e} % provides \textsubscript
\ifnum 0\ifxetex 1\fi\ifluatex 1\fi=0 % if pdftex
  \usepackage[T1]{fontenc}
  \usepackage[utf8]{inputenc}
\else % if luatex or xelatex
  \ifxetex
    \usepackage{mathspec}
  \else
    \usepackage{fontspec}
  \fi
  \defaultfontfeatures{Ligatures=TeX,Scale=MatchLowercase}
\fi
% use upquote if available, for straight quotes in verbatim environments
\IfFileExists{upquote.sty}{\usepackage{upquote}}{}
% use microtype if available
\IfFileExists{microtype.sty}{%
\usepackage{microtype}
\UseMicrotypeSet[protrusion]{basicmath} % disable protrusion for tt fonts
}{}
\usepackage[margin=1in]{geometry}
\usepackage{hyperref}
\hypersetup{unicode=true,
            pdftitle={Data Wrangling and Visualization},
            pdfauthor={Thomas Allen},
            pdfborder={0 0 0},
            breaklinks=true}
\urlstyle{same}  % don't use monospace font for urls
\usepackage{color}
\usepackage{fancyvrb}
\newcommand{\VerbBar}{|}
\newcommand{\VERB}{\Verb[commandchars=\\\{\}]}
\DefineVerbatimEnvironment{Highlighting}{Verbatim}{commandchars=\\\{\}}
% Add ',fontsize=\small' for more characters per line
\usepackage{framed}
\definecolor{shadecolor}{RGB}{248,248,248}
\newenvironment{Shaded}{\begin{snugshade}}{\end{snugshade}}
\newcommand{\AlertTok}[1]{\textcolor[rgb]{0.94,0.16,0.16}{#1}}
\newcommand{\AnnotationTok}[1]{\textcolor[rgb]{0.56,0.35,0.01}{\textbf{\textit{#1}}}}
\newcommand{\AttributeTok}[1]{\textcolor[rgb]{0.77,0.63,0.00}{#1}}
\newcommand{\BaseNTok}[1]{\textcolor[rgb]{0.00,0.00,0.81}{#1}}
\newcommand{\BuiltInTok}[1]{#1}
\newcommand{\CharTok}[1]{\textcolor[rgb]{0.31,0.60,0.02}{#1}}
\newcommand{\CommentTok}[1]{\textcolor[rgb]{0.56,0.35,0.01}{\textit{#1}}}
\newcommand{\CommentVarTok}[1]{\textcolor[rgb]{0.56,0.35,0.01}{\textbf{\textit{#1}}}}
\newcommand{\ConstantTok}[1]{\textcolor[rgb]{0.00,0.00,0.00}{#1}}
\newcommand{\ControlFlowTok}[1]{\textcolor[rgb]{0.13,0.29,0.53}{\textbf{#1}}}
\newcommand{\DataTypeTok}[1]{\textcolor[rgb]{0.13,0.29,0.53}{#1}}
\newcommand{\DecValTok}[1]{\textcolor[rgb]{0.00,0.00,0.81}{#1}}
\newcommand{\DocumentationTok}[1]{\textcolor[rgb]{0.56,0.35,0.01}{\textbf{\textit{#1}}}}
\newcommand{\ErrorTok}[1]{\textcolor[rgb]{0.64,0.00,0.00}{\textbf{#1}}}
\newcommand{\ExtensionTok}[1]{#1}
\newcommand{\FloatTok}[1]{\textcolor[rgb]{0.00,0.00,0.81}{#1}}
\newcommand{\FunctionTok}[1]{\textcolor[rgb]{0.00,0.00,0.00}{#1}}
\newcommand{\ImportTok}[1]{#1}
\newcommand{\InformationTok}[1]{\textcolor[rgb]{0.56,0.35,0.01}{\textbf{\textit{#1}}}}
\newcommand{\KeywordTok}[1]{\textcolor[rgb]{0.13,0.29,0.53}{\textbf{#1}}}
\newcommand{\NormalTok}[1]{#1}
\newcommand{\OperatorTok}[1]{\textcolor[rgb]{0.81,0.36,0.00}{\textbf{#1}}}
\newcommand{\OtherTok}[1]{\textcolor[rgb]{0.56,0.35,0.01}{#1}}
\newcommand{\PreprocessorTok}[1]{\textcolor[rgb]{0.56,0.35,0.01}{\textit{#1}}}
\newcommand{\RegionMarkerTok}[1]{#1}
\newcommand{\SpecialCharTok}[1]{\textcolor[rgb]{0.00,0.00,0.00}{#1}}
\newcommand{\SpecialStringTok}[1]{\textcolor[rgb]{0.31,0.60,0.02}{#1}}
\newcommand{\StringTok}[1]{\textcolor[rgb]{0.31,0.60,0.02}{#1}}
\newcommand{\VariableTok}[1]{\textcolor[rgb]{0.00,0.00,0.00}{#1}}
\newcommand{\VerbatimStringTok}[1]{\textcolor[rgb]{0.31,0.60,0.02}{#1}}
\newcommand{\WarningTok}[1]{\textcolor[rgb]{0.56,0.35,0.01}{\textbf{\textit{#1}}}}
\usepackage{longtable,booktabs}
\usepackage{graphicx,grffile}
\makeatletter
\def\maxwidth{\ifdim\Gin@nat@width>\linewidth\linewidth\else\Gin@nat@width\fi}
\def\maxheight{\ifdim\Gin@nat@height>\textheight\textheight\else\Gin@nat@height\fi}
\makeatother
% Scale images if necessary, so that they will not overflow the page
% margins by default, and it is still possible to overwrite the defaults
% using explicit options in \includegraphics[width, height, ...]{}
\setkeys{Gin}{width=\maxwidth,height=\maxheight,keepaspectratio}
\IfFileExists{parskip.sty}{%
\usepackage{parskip}
}{% else
\setlength{\parindent}{0pt}
\setlength{\parskip}{6pt plus 2pt minus 1pt}
}
\setlength{\emergencystretch}{3em}  % prevent overfull lines
\providecommand{\tightlist}{%
  \setlength{\itemsep}{0pt}\setlength{\parskip}{0pt}}
\setcounter{secnumdepth}{0}
% Redefines (sub)paragraphs to behave more like sections
\ifx\paragraph\undefined\else
\let\oldparagraph\paragraph
\renewcommand{\paragraph}[1]{\oldparagraph{#1}\mbox{}}
\fi
\ifx\subparagraph\undefined\else
\let\oldsubparagraph\subparagraph
\renewcommand{\subparagraph}[1]{\oldsubparagraph{#1}\mbox{}}
\fi

%%% Use protect on footnotes to avoid problems with footnotes in titles
\let\rmarkdownfootnote\footnote%
\def\footnote{\protect\rmarkdownfootnote}

%%% Change title format to be more compact
\usepackage{titling}

% Create subtitle command for use in maketitle
\newcommand{\subtitle}[1]{
  \posttitle{
    \begin{center}\large#1\end{center}
    }
}

\setlength{\droptitle}{-2em}

  \title{Data Wrangling and Visualization}
    \pretitle{\vspace{\droptitle}\centering\huge}
  \posttitle{\par}
    \author{Thomas Allen}
    \preauthor{\centering\large\emph}
  \postauthor{\par}
      \predate{\centering\large\emph}
  \postdate{\par}
    \date{9/30/2020}


\begin{document}
\maketitle

\begin{Shaded}
\begin{Highlighting}[]
\NormalTok{within_radius_of_point <-}\StringTok{ }\ControlFlowTok{function}\NormalTok{(df,point,radius) \{}
  
  \CommentTok{# df : dataframe with RA and Dec columns (Decimal Degrees)}
  
  \CommentTok{# point : 2-element vector with center point RA and Dec (Decimal Degrees)}
  
  \CommentTok{# radius : radius to search within (Arcseconds)}
  
  
\NormalTok{df <-}\StringTok{  }\NormalTok{df }\OperatorTok\StringTok{ }
\StringTok{  }
\StringTok{  }\KeywordTok{mutate}\NormalTok{(}\DataTypeTok{new_column_distance =} \KeywordTok{sqrt}\NormalTok{( ((point[}\DecValTok{1}\NormalTok{] }\OperatorTok{-}\StringTok{ }\NormalTok{RA) }\OperatorTok{*}\StringTok{ }\FloatTok{3600.} \OperatorTok{*}\StringTok{ }\KeywordTok{cos}\NormalTok{((pi}\OperatorTok{/}\FloatTok{180.}\NormalTok{)}\OperatorTok{*}\NormalTok{Dec))}\OperatorTok{^}\DecValTok{2}   \OperatorTok{+}\StringTok{  }\NormalTok{((Dec}\OperatorTok{-}\NormalTok{point[}\DecValTok{2}\NormalTok{]) }\OperatorTok{*}\StringTok{ }\FloatTok{3600.}\NormalTok{)}\OperatorTok{^}\DecValTok{2}\NormalTok{)) }\OperatorTok\StringTok{ }
\StringTok{  }\KeywordTok{mutate}\NormalTok{(}\DataTypeTok{new_column =} \KeywordTok{if_else}\NormalTok{(new_column_distance }\OperatorTok{<}\StringTok{ }\NormalTok{radius,}\StringTok{"yes"}\NormalTok{, }\StringTok{"no"}\NormalTok{))}
  
\NormalTok{df}
  
\NormalTok{\}}
\end{Highlighting}
\end{Shaded}

\begin{Shaded}
\begin{Highlighting}[]
\NormalTok{df <-}\StringTok{ }\KeywordTok{read_csv}\NormalTok{(}\StringTok{"~/Websites/Working/thomas-s-allen 2/cep_tutorial/data/J.ApJ.750.125.ysos_2MASS_GAIA_raw.csv"}\NormalTok{)}
\end{Highlighting}
\end{Shaded}

\begin{verbatim}
## Warning: Duplicated column names deduplicated: 'angDist' =>
## 'angDist_1' [2], 'RAJ2000' => 'RAJ2000_1' [24], 'DEJ2000' =>
## 'DEJ2000_1' [25], 'Jmag' => 'Jmag_1' [29], 'Hmag' => 'Hmag_1' [30],
## 'errHalfMaj' => 'errHalfMaj_1' [41], 'errHalfMin' => 'errHalfMin_1' [42],
## 'errPosAng' => 'errPosAng_1' [43]
\end{verbatim}

\begin{verbatim}
## Parsed with column specification:
## cols(
##   .default = col_double(),
##   RAJ2000 = col_character(),
##   DEJ2000 = col_character(),
##   Cl = col_character(),
##   Note = col_character(),
##   M07 = col_integer(),
##   `2M` = col_character(),
##   Simbad = col_character(),
##   `2MASS` = col_character(),
##   errPosAng = col_integer(),
##   Qfl = col_character(),
##   Rfl = col_integer(),
##   X = col_integer(),
##   errPosAng_1 = col_integer(),
##   duplicated_source = col_integer(),
##   rv_nb_transits = col_integer()
## )
\end{verbatim}

\begin{verbatim}
## See spec(...) for full column specifications.
\end{verbatim}

\begin{Shaded}
\begin{Highlighting}[]
\KeywordTok{nrow}\NormalTok{(df)}
\end{Highlighting}
\end{Shaded}

\begin{verbatim}
## [1] 2569
\end{verbatim}

\begin{Shaded}
\begin{Highlighting}[]
\CommentTok{#east_center <- c(344.2083, 62.6656)}
\NormalTok{east_center <-}\StringTok{ }\KeywordTok{c}\NormalTok{(}\FloatTok{344.2060000}\NormalTok{, }\FloatTok{62.6654278}\NormalTok{)}
\NormalTok{west_center <-}\StringTok{ }\KeywordTok{c}\NormalTok{(}\FloatTok{343.4625}\NormalTok{, }\FloatTok{62.5936}\NormalTok{)}

\NormalTok{radius <-}\StringTok{ }\FloatTok{300.}

\NormalTok{df }\OperatorTok\StringTok{ }\KeywordTok{distinct}\NormalTok{(Note)}
\end{Highlighting}
\end{Shaded}

\begin{verbatim}
## # A tibble: 2 x 1
##   Note 
##   <chr>
## 1 X-ray
## 2 <NA>
\end{verbatim}

\begin{Shaded}
\begin{Highlighting}[]
\NormalTok{parallax_conversion <-}\StringTok{ }\DecValTok{10}\OperatorTok{^}\DecValTok{3} \CommentTok{# milli arcseconds to arcseconds}

\NormalTok{cols_to_keep <-}\StringTok{ }\KeywordTok{c}\NormalTok{(}
                  \StringTok{"RA"}\NormalTok{,}
                  \StringTok{"Dec"}\NormalTok{,}
                  \StringTok{"Type"}\NormalTok{,}
                  \StringTok{"Band1"}\NormalTok{,}
                  \StringTok{"Band2"}\NormalTok{,}
                  \StringTok{"Band3"}\NormalTok{,}
                  \StringTok{"Band4"}\NormalTok{,}
                  \StringTok{"Band5"}\NormalTok{,}
                  \StringTok{"Band6"}\NormalTok{,}
                  \StringTok{"Band7"}\NormalTok{,}
                  \StringTok{"Band8"}\NormalTok{,}
                  \StringTok{"Band9"}\NormalTok{,}
                  \StringTok{"Band10"}\NormalTok{,}
                  \StringTok{"Dust"}\NormalTok{,}
                  \StringTok{"Xray"}\NormalTok{,}
                  \StringTok{"parallax"}\NormalTok{,}
                  \StringTok{"parallax_error"}\NormalTok{,}
                  \StringTok{"pmra"}\NormalTok{,}
                  \StringTok{"pmra_error"}\NormalTok{,}
                  \StringTok{"pmdec"}\NormalTok{,}
                  \StringTok{"pmdec_error"}\NormalTok{,}
                  \StringTok{"teff_val"}\NormalTok{,}
                  \StringTok{"a_g_val"}\NormalTok{,}
                  \StringTok{"radius_val"}\NormalTok{,}
                  \StringTok{"lum_val"}
\NormalTok{                  )}


\NormalTok{df <-}\StringTok{ }\NormalTok{df }\OperatorTok
\StringTok{  }\KeywordTok{mutate}\NormalTok{(}\StringTok{`}\DataTypeTok{V-I}\StringTok{`}\NormalTok{ =}\StringTok{ }\NormalTok{Vmag}\OperatorTok{-}\StringTok{`}\DataTypeTok{V-I}\StringTok{`}\NormalTok{) }\OperatorTok\StringTok{ }
\StringTok{  }\KeywordTok{mutate}\NormalTok{(}\DataTypeTok{Cl =} \KeywordTok{fct_recode}\NormalTok{(Cl,}\StringTok{"no disk"}\NormalTok{=}\StringTok{"III"}\NormalTok{,}\StringTok{"disk"}\NormalTok{=}\StringTok{"II"}\NormalTok{,}\StringTok{"disk"}\NormalTok{=}\StringTok{"TD"}\NormalTok{,}\StringTok{"envelope"}\NormalTok{=}\StringTok{"I"}\NormalTok{)) }\OperatorTok\StringTok{ }
\StringTok{  }\KeywordTok{mutate}\NormalTok{(}\DataTypeTok{Note =} \KeywordTok{case_when}\NormalTok{(Note}\OperatorTok{==}\StringTok{"x-ray"} \OperatorTok{~}\StringTok{ "yes"}\NormalTok{,}
                            \KeywordTok{is.na}\NormalTok{(Note)}\OperatorTok{==}\OtherTok{TRUE} \OperatorTok{~}\StringTok{ "no"}\NormalTok{)) }\OperatorTok\StringTok{ }
\StringTok{  }\KeywordTok{rename}\NormalTok{(}
    \DataTypeTok{RA =} \StringTok{`}\DataTypeTok{_RAJ2000}\StringTok{`}\NormalTok{,}
    \DataTypeTok{Dec =} \StringTok{`}\DataTypeTok{_DEJ2000}\StringTok{`}\NormalTok{,}
    \DataTypeTok{Type =}\NormalTok{ Cl,}
    \DataTypeTok{Band1 =}\NormalTok{ Vmag,}
    \DataTypeTok{Band2 =} \StringTok{`}\DataTypeTok{V-I}\StringTok{`}\NormalTok{,}
    \DataTypeTok{Band3 =}\NormalTok{ Jmag,}
    \DataTypeTok{Band4 =}\NormalTok{ Hmag,}
    \DataTypeTok{Band5 =}\NormalTok{ Ksmag,}
    \DataTypeTok{Band6 =} \StringTok{`}\DataTypeTok{[3.6]}\StringTok{`}\NormalTok{,}
    \DataTypeTok{Band7 =} \StringTok{`}\DataTypeTok{[4.5]}\StringTok{`}\NormalTok{,}
    \DataTypeTok{Band8 =} \StringTok{`}\DataTypeTok{[5.8]}\StringTok{`}\NormalTok{,}
    \DataTypeTok{Band9 =} \StringTok{`}\DataTypeTok{[8.0]}\StringTok{`}\NormalTok{,}
    \DataTypeTok{Band10 =} \StringTok{`}\DataTypeTok{[24]}\StringTok{`}\NormalTok{,}
    \DataTypeTok{Xray =}\NormalTok{ Note,}
    \DataTypeTok{Dust =}\NormalTok{ AKs}
\NormalTok{  ) }\OperatorTok\StringTok{ }
\StringTok{  }\KeywordTok{select}\NormalTok{(cols_to_keep) }\OperatorTok\StringTok{ }
\StringTok{  }\CommentTok{#filter(Type %in% c("Disk","No Disk")) %>% }
\StringTok{  }\KeywordTok{mutate}\NormalTok{(}\DataTypeTok{Distance =} \DecValTok{1}\OperatorTok{/}\StringTok{ }\NormalTok{(parallax}\OperatorTok{/}\NormalTok{parallax_conversion)) }\OperatorTok\StringTok{ }
\StringTok{  }\KeywordTok{filter}\NormalTok{(Distance }\OperatorTok{>}\StringTok{ }\DecValTok{0}\NormalTok{,}
\NormalTok{         Distance }\OperatorTok{<}\StringTok{ }\DecValTok{10000}\NormalTok{) }\OperatorTok\StringTok{ }
\StringTok{  }\KeywordTok{within_radius_of_point}\NormalTok{(east_center,radius) }\OperatorTok\StringTok{ }
\StringTok{  }\KeywordTok{mutate}\NormalTok{(}\DataTypeTok{east_cluster =}\NormalTok{ new_column) }\OperatorTok\StringTok{ }
\StringTok{  }\KeywordTok{select}\NormalTok{(}\OperatorTok{-}\NormalTok{new_column) }\OperatorTok\StringTok{ }
\StringTok{  }\KeywordTok{within_radius_of_point}\NormalTok{(west_center,radius) }\OperatorTok\StringTok{ }
\StringTok{  }\KeywordTok{mutate}\NormalTok{(}\DataTypeTok{west_cluster =}\NormalTok{ new_column) }\OperatorTok\StringTok{ }
\StringTok{  }\KeywordTok{select}\NormalTok{(}\OperatorTok{-}\NormalTok{new_column) }\OperatorTok\StringTok{ }
\StringTok{  }\KeywordTok{mutate}\NormalTok{(}\DataTypeTok{Region =} \KeywordTok{if_else}\NormalTok{(east_cluster}\OperatorTok{==}\StringTok{"yes"}\NormalTok{,}\StringTok{"east"}\NormalTok{,}\KeywordTok{if_else}\NormalTok{(west_cluster}\OperatorTok{==}\StringTok{"yes"}\NormalTok{,}\StringTok{"west"}\NormalTok{,}\StringTok{"halo"}\NormalTok{))) }\OperatorTok\StringTok{ }
\StringTok{  }\KeywordTok{select}\NormalTok{(}\OperatorTok{-}\NormalTok{new_column_distance,}\OperatorTok{-}\NormalTok{east_cluster,}\OperatorTok{-}\NormalTok{west_cluster)}

\NormalTok{df}
\end{Highlighting}
\end{Shaded}

\begin{verbatim}
## # A tibble: 2,166 x 27
##       RA   Dec Type   Band1 Band2 Band3 Band4 Band5 Band6 Band7 Band8 Band9
##    <dbl> <dbl> <fct>  <dbl> <dbl> <dbl> <dbl> <dbl> <dbl> <dbl> <dbl> <dbl>
##  1  344.  62.6 no di~  15.1  13.3  12.1  11.6  11.4  11.2  11.2  11.2  11.2
##  2  344.  62.8 no di~  17.6  15.2  13.8  13.2  12.9  12.7  12.7  12.6  12.7
##  3  344.  62.7 no di~  15.9  13.8  12.0  11.2  10.9  10.8  10.8  10.7  10.7
##  4  344.  62.6 no di~  20.2  17.1  15.0  14.1  13.7  13.4  13.4  13.2  13.3
##  5  344.  62.5 no di~  18.8  16.1  14.1  13.2  12.9  12.7  12.6  12.6  12.5
##  6  344.  62.4 no di~  19.6  16.4  14.2  13.3  13.0  12.6  12.6  12.4  11.8
##  7  344.  62.6 no di~  22.2  18.1  15.8  14.8  14.3  14.0  13.9  14.0  13.9
##  8  344.  62.5 no di~  19.4  15.9  13.2  12.0  11.6  11.2  11.2  11.1  11.2
##  9  344.  62.4 no di~  21.2  17.5  14.7  13.5  13.1  12.7  12.7  12.7  12.6
## 10  344.  62.5 no di~  20.1  16.9  14.6  13.5  13.2  13.0  12.9  12.9  13.0
## # ... with 2,156 more rows, and 15 more variables: Band10 <dbl>,
## #   Dust <dbl>, Xray <chr>, parallax <dbl>, parallax_error <dbl>,
## #   pmra <dbl>, pmra_error <dbl>, pmdec <dbl>, pmdec_error <dbl>,
## #   teff_val <dbl>, a_g_val <dbl>, radius_val <dbl>, lum_val <dbl>,
## #   Distance <dbl>, Region <chr>
\end{verbatim}

\begin{Shaded}
\begin{Highlighting}[]
\CommentTok{#write_csv(df, file="/Desktop/exam_prep.csv")}
\end{Highlighting}
\end{Shaded}

\hypertarget{problem-1}{%
\subsection{Problem 1}\label{problem-1}}

We are interested in exploring whether the presence (or lack thereof) of
a protoplanetary disk around a young star depends on the location of
that star in its birth cluster. In this dataset, the \texttt{Region}
variable gives the location of the stars in the cluster as ``east'',
``west'' and ``halo''. And the \texttt{Type} variable states disk
presence as ``disk'', ``no disk'' and ``envelope''. You can either
include cases of \texttt{Type} ``envelope'' with the ``disk'' cases or
ignore them. For this problem:

\begin{itemize}
\tightlist
\item
  Perform an exploratory data analysis, choosing an appropriate
  visualization and table.
\item
  Consider two cases, first, just the ``east'' and ``west'' categories
  in \texttt{Region} and second, all three categories in
  \texttt{Region}: ``east,''west" and ``halo''. By both choosing an
  appropriate probability model and using a computational method with
  \texttt{infer()}, for each case:

  \begin{itemize}
  \tightlist
  \item
    Calculate an appropriate point estimate and construct a
    \(95\%\)-confidence interval around your point estimate. For the
    case of more than two categories, discuss the appropriatness of a
    confidence interval.
  \item
    Formulate null and alternative hypotheses and construct a hypothesis
    test, calculate a p-value and draw a conclusion.
  \item
    Discuss the relevent criteria for each approach and whether the data
    conform to the criteria.
  \end{itemize}
\end{itemize}

\textbf{Note:} After using the \texttt{filter()} function, it may be
useful to use \texttt{fct\_drop()} to ensure any categories with a 0
count are dropped.

\begin{center}\rule{0.5\linewidth}{\linethickness}\end{center}

\hypertarget{exploratory-data-analysis}{%
\subsubsection{Exploratory Data
Analysis}\label{exploratory-data-analysis}}

First, we want to explore our data. Create a useful table and
visualization to show the relationship between \texttt{Type} and
\texttt{Region}.

\begin{Shaded}
\begin{Highlighting}[]
\NormalTok{df_counts <-}\StringTok{ }\NormalTok{df }\OperatorTok\StringTok{ }
\StringTok{ }\CommentTok{# filter(Xray=="yes",}
\StringTok{ }\CommentTok{#       Type %in% c("Disk","No Disk")) %>% }
\StringTok{  }\KeywordTok{filter}\NormalTok{(Type }\OperatorTok\StringTok{ }\KeywordTok{c}\NormalTok{(}\StringTok{"disk"}\NormalTok{,}\StringTok{"no disk"}\NormalTok{)) }\OperatorTok\StringTok{ }
\StringTok{         }\CommentTok{#Subcluster %in% c("east","west")) %>% }
\StringTok{  }\KeywordTok{group_by}\NormalTok{(Region,Type) }\OperatorTok\StringTok{ }
\StringTok{  }\KeywordTok{summarize}\NormalTok{(}\DataTypeTok{counts=}\KeywordTok{n}\NormalTok{()) }\OperatorTok\StringTok{ }
\StringTok{  }\KeywordTok{mutate}\NormalTok{(}\DataTypeTok{freq=}\NormalTok{counts}\OperatorTok{/}\KeywordTok{sum}\NormalTok{(counts))}

\NormalTok{df_counts}
\end{Highlighting}
\end{Shaded}

\begin{verbatim}
## # A tibble: 6 x 4
## # Groups:   Region [3]
##   Region Type    counts  freq
##   <chr>  <fct>    <int> <dbl>
## 1 east   disk       192 0.444
## 2 east   no disk    240 0.556
## 3 halo   disk       507 0.353
## 4 halo   no disk    929 0.647
## 5 west   disk       159 0.543
## 6 west   no disk    134 0.457
\end{verbatim}

\begin{Shaded}
\begin{Highlighting}[]
\NormalTok{df_counts }\OperatorTok\StringTok{ }
\StringTok{  }\KeywordTok{ggplot}\NormalTok{(}\KeywordTok{aes}\NormalTok{(}\DataTypeTok{x=}\NormalTok{Region,}\DataTypeTok{y=}\NormalTok{counts,}\DataTypeTok{fill=}\NormalTok{Type)) }\OperatorTok{+}
\StringTok{    }\KeywordTok{geom_col}\NormalTok{(}\DataTypeTok{position=}\StringTok{"fill"}\NormalTok{)}
\end{Highlighting}
\end{Shaded}

\includegraphics{inference_exam_prep_files/figure-latex/unnamed-chunk-3-1.pdf}

\hypertarget{two-categories}{%
\subsubsection{Two Categories}\label{two-categories}}

\hypertarget{confidence-intervals}{%
\paragraph{Confidence intervals}\label{confidence-intervals}}

Now just consider the regions ``east'' and ``west''. For eash region
estimate the proportion of stars with a protoplanetary disk and
determine \(95\%\) confidence intervals.

\begin{center}\rule{0.5\linewidth}{\linethickness}\end{center}

\hypertarget{approximation-with-probability-models}{%
\subparagraph{Approximation with probability
models}\label{approximation-with-probability-models}}

The confidence interval around a single proportion point estimate is \[
\begin{aligned}
\hat{p} & \pm MOE \\
\hat{p} & \pm z^{*} \times \sqrt{\frac{\hat{p}(1-\hat{p})}{n}}
\end{aligned}
\] For the ``east'' region, since there are 192 stars with a disk and
240 without, \(\hat{p} = \frac{192}{192 + 240} = 0.44\): \[
\begin{aligned}
\hat{p_{east}} & \pm z^{*} \times \sqrt{\frac{\hat{p}(1-\hat{p})}{n}} \\
0.44 & \pm 1.96 \times \sqrt{\frac{0.44(1-0.44)}{432}} \\
0.44 & \pm 0.05
\end{aligned}
\]

\begin{Shaded}
\begin{Highlighting}[]
\FloatTok{1.96} \OperatorTok{*}\StringTok{ }\KeywordTok{sqrt}\NormalTok{( (}\FloatTok{0.44} \OperatorTok{*}\StringTok{ }\NormalTok{(}\DecValTok{1} \OperatorTok{-}\StringTok{ }\FloatTok{0.44}\NormalTok{)) }\OperatorTok{/}\StringTok{ }\DecValTok{432}\NormalTok{)}
\end{Highlighting}
\end{Shaded}

\begin{verbatim}
## [1] 0.04680956
\end{verbatim}

And for the ``west'' region, since there are 159 stars with a disk and
134 without, \(\hat{p} = \frac{159}{159 + 134} = 0.54\): \[
\begin{aligned}
\hat{p_{west}} & \pm z^{*} \times \sqrt{\frac{\hat{p}(1-\hat{p})}{n}} \\
0.54 & \pm 1.96 \times \sqrt{\frac{0.54(1-0.54)}{293}} \\
0.54 & \pm 0.06
\end{aligned}
\]

\begin{Shaded}
\begin{Highlighting}[]
\FloatTok{1.96} \OperatorTok{*}\StringTok{ }\KeywordTok{sqrt}\NormalTok{( (}\FloatTok{0.54} \OperatorTok{*}\StringTok{ }\NormalTok{(}\DecValTok{1} \OperatorTok{-}\StringTok{ }\FloatTok{0.54}\NormalTok{)) }\OperatorTok{/}\StringTok{ }\DecValTok{293}\NormalTok{)}
\end{Highlighting}
\end{Shaded}

\begin{verbatim}
## [1] 0.05706871
\end{verbatim}

Therefore, we think that a plausable range of values that contain the
popultation porportion in the east region is, \(p_{east}\), is
\(0.39 - 0.49\), and for the west, \(p_{west}\), is \(0.48 - 0.6\).

\hypertarget{computational-method-with-infer}{%
\subparagraph{Computational method with
Infer}\label{computational-method-with-infer}}

We will use \texttt{infer} to construct bootstrap estimations of the
sampling distribution.

\begin{Shaded}
\begin{Highlighting}[]
\CommentTok{# First }

\NormalTok{df_two <-}\StringTok{ }\NormalTok{df }\OperatorTok\StringTok{ }
\StringTok{  }\KeywordTok{filter}\NormalTok{(Region }\OperatorTok\StringTok{ }\KeywordTok{c}\NormalTok{(}\StringTok{"east"}\NormalTok{,}\StringTok{"west"}\NormalTok{),}
\NormalTok{         Type }\OperatorTok\StringTok{ }\KeywordTok{c}\NormalTok{(}\StringTok{"disk"}\NormalTok{,}\StringTok{"no disk"}\NormalTok{)) }\OperatorTok\StringTok{ }
\StringTok{  }\KeywordTok{mutate}\NormalTok{(}\DataTypeTok{Region=}\KeywordTok{factor}\NormalTok{(Region),}
         \DataTypeTok{Type=}\KeywordTok{factor}\NormalTok{(Type))}

\NormalTok{df_two }\OperatorTok\StringTok{ }
\StringTok{  }\KeywordTok{group_by}\NormalTok{(Region, Type) }\OperatorTok
\StringTok{  }\KeywordTok{summarise}\NormalTok{(}\DataTypeTok{count =} \KeywordTok{n}\NormalTok{()) }\OperatorTok\StringTok{ }
\StringTok{  }\KeywordTok{mutate}\NormalTok{(}\DataTypeTok{Type =} \KeywordTok{fct_drop}\NormalTok{(Type))}
\end{Highlighting}
\end{Shaded}

\begin{verbatim}
## # A tibble: 4 x 3
## # Groups:   Region [2]
##   Region Type    count
##   <fct>  <fct>   <int>
## 1 east   disk      192
## 2 east   no disk   240
## 3 west   disk      159
## 4 west   no disk   134
\end{verbatim}

\begin{Shaded}
\begin{Highlighting}[]
\CommentTok{# east}
\NormalTok{prop_east <-}\StringTok{ }\NormalTok{df_two }\OperatorTok\StringTok{ }
\StringTok{  }\KeywordTok{filter}\NormalTok{(Region }\OperatorTok{==}\StringTok{ "east"}\NormalTok{) }\OperatorTok\StringTok{ }
\StringTok{  }\KeywordTok{specify}\NormalTok{(}\DataTypeTok{response =}\NormalTok{ Type, }\DataTypeTok{success =} \StringTok{"disk"}\NormalTok{) }\OperatorTok\StringTok{ }
\StringTok{  }\KeywordTok{calculate}\NormalTok{(}\DataTypeTok{stat =} \StringTok{"prop"}\NormalTok{)}

\NormalTok{boot_east <-}\StringTok{ }\NormalTok{df_two }\OperatorTok\StringTok{ }
\StringTok{  }\KeywordTok{filter}\NormalTok{(Region }\OperatorTok{==}\StringTok{ "east"}\NormalTok{) }\OperatorTok\StringTok{ }
\StringTok{  }\KeywordTok{specify}\NormalTok{(}\DataTypeTok{response =}\NormalTok{ Type, }\DataTypeTok{success =} \StringTok{"disk"}\NormalTok{) }\OperatorTok\StringTok{ }
\StringTok{  }\KeywordTok{generate}\NormalTok{(}\DataTypeTok{reps=}\DecValTok{1000}\NormalTok{,}\DataTypeTok{type=}\StringTok{"bootstrap"}\NormalTok{) }\OperatorTok\StringTok{ }
\StringTok{  }\KeywordTok{calculate}\NormalTok{(}\DataTypeTok{stat =} \StringTok{"prop"}\NormalTok{) }

\NormalTok{ci_east <-}\StringTok{ }\NormalTok{boot_east }\OperatorTok\StringTok{ }
\StringTok{  }\KeywordTok{get_confidence_interval}\NormalTok{(}\DataTypeTok{type =} \StringTok{"se"}\NormalTok{, }\DataTypeTok{point_estimate =}\NormalTok{ prop_east)}
\end{Highlighting}
\end{Shaded}

\begin{verbatim}
## Using `level = 0.95` to compute confidence interval.
\end{verbatim}

\begin{Shaded}
\begin{Highlighting}[]
\NormalTok{prop_east}
\end{Highlighting}
\end{Shaded}

\begin{verbatim}
## # A tibble: 1 x 1
##    stat
##   <dbl>
## 1 0.444
\end{verbatim}

\begin{Shaded}
\begin{Highlighting}[]
\NormalTok{ci_east}
\end{Highlighting}
\end{Shaded}

\begin{verbatim}
## # A tibble: 1 x 2
##   lower_ci upper_ci
##      <dbl>    <dbl>
## 1    0.397    0.492
\end{verbatim}

\begin{Shaded}
\begin{Highlighting}[]
\CommentTok{# west}
\NormalTok{prop_west <-}\StringTok{ }\NormalTok{df_two }\OperatorTok\StringTok{ }
\StringTok{  }\KeywordTok{filter}\NormalTok{(Region }\OperatorTok{==}\StringTok{ "west"}\NormalTok{) }\OperatorTok\StringTok{ }
\StringTok{  }\KeywordTok{specify}\NormalTok{(}\DataTypeTok{response =}\NormalTok{ Type, }\DataTypeTok{success =} \StringTok{"disk"}\NormalTok{) }\OperatorTok\StringTok{ }
\StringTok{  }\KeywordTok{calculate}\NormalTok{(}\DataTypeTok{stat =} \StringTok{"prop"}\NormalTok{)}

\NormalTok{boot_west <-}\StringTok{ }\NormalTok{df_two }\OperatorTok\StringTok{ }
\StringTok{  }\KeywordTok{filter}\NormalTok{(Region }\OperatorTok{==}\StringTok{ "west"}\NormalTok{) }\OperatorTok\StringTok{ }
\StringTok{  }\KeywordTok{specify}\NormalTok{(}\DataTypeTok{response =}\NormalTok{ Type, }\DataTypeTok{success =} \StringTok{"disk"}\NormalTok{) }\OperatorTok\StringTok{ }
\StringTok{  }\KeywordTok{generate}\NormalTok{(}\DataTypeTok{reps=}\DecValTok{1000}\NormalTok{,}\DataTypeTok{type=}\StringTok{"bootstrap"}\NormalTok{) }\OperatorTok\StringTok{ }
\StringTok{  }\KeywordTok{calculate}\NormalTok{(}\DataTypeTok{stat =} \StringTok{"prop"}\NormalTok{) }

\NormalTok{ci_west <-}\StringTok{ }\NormalTok{boot_west }\OperatorTok\StringTok{ }
\StringTok{  }\KeywordTok{get_confidence_interval}\NormalTok{(}\DataTypeTok{type =} \StringTok{"se"}\NormalTok{, }\DataTypeTok{point_estimate =}\NormalTok{ prop_west)}
\end{Highlighting}
\end{Shaded}

\begin{verbatim}
## Using `level = 0.95` to compute confidence interval.
\end{verbatim}

\begin{Shaded}
\begin{Highlighting}[]
\NormalTok{prop_west}
\end{Highlighting}
\end{Shaded}

\begin{verbatim}
## # A tibble: 1 x 1
##    stat
##   <dbl>
## 1 0.543
\end{verbatim}

\begin{Shaded}
\begin{Highlighting}[]
\NormalTok{ci_west}
\end{Highlighting}
\end{Shaded}

\begin{verbatim}
## # A tibble: 1 x 2
##   lower_ci upper_ci
##      <dbl>    <dbl>
## 1    0.488    0.598
\end{verbatim}

\begin{Shaded}
\begin{Highlighting}[]
\CommentTok{# Construct Bootstrap Distributions}

\CommentTok{#boot_dist <- df_two %>% }
\CommentTok{#  specify(response = Type, success = "disk") %>% }
\CommentTok{#  generate(reps=1000, type="bootstrap") %>% }
\CommentTok{#  calculate(stat = "prop")}
\end{Highlighting}
\end{Shaded}

\begin{center}\rule{0.5\linewidth}{\linethickness}\end{center}

\hypertarget{hypothesis-testing}{%
\subparagraph{Hypothesis testing}\label{hypothesis-testing}}

Now answer the question, ``Do the proportions of stars with disks differ
with region?'' Formulate a null and alterntive hypothesis, calculate a
test statistic and a p-value and then make a conclusion about your
hypotheis.

We will formulate our hypotheses as: \$\$
\textbackslash{}begin\{aligned\} H\_\{0\} \&: p\_\{east\} - p\_\{west\}
= 0 \textbackslash{} H\_\{A\} \&: p\_\{east\} - p\_\{west\} \ne 0

\textbackslash{}end\{aligned\} \$\$

We can calculate a \(z-score\) test statistic to test this hypothesis:
\[
\begin{aligned}
z & = \frac{\hat{p_{east}} - \hat{p_{west}} - 0}{\sqrt{\frac{\hat{p_{east}}(1-\hat{p_{east}})}{n_{east}} + 
    \frac{\hat{p_{west}}(1-\hat{p_{west}})}{n_{west}}}} \\
z & = \frac{0.44 - 0.54 - 0}{\sqrt{\frac{0.44(1-0.44)}{432} + 
    \frac{0.54(1-0.54)}{293}}} \\
z & = -2.66
\end{aligned}
\]

\begin{Shaded}
\begin{Highlighting}[]
\NormalTok{z_score <-}\StringTok{ }\NormalTok{(}\FloatTok{0.44} \OperatorTok{-}\StringTok{ }\FloatTok{0.54} \OperatorTok{-}\StringTok{ }\DecValTok{0}\NormalTok{) }\OperatorTok{/}\StringTok{ }\KeywordTok{sqrt}\NormalTok{( (}\FloatTok{0.44}\OperatorTok{*}\NormalTok{(}\DecValTok{1}\FloatTok{-0.44}\NormalTok{)}\OperatorTok{/}\DecValTok{432}\NormalTok{) }\OperatorTok{+}\StringTok{ }\NormalTok{(}\FloatTok{0.54}\OperatorTok{*}\NormalTok{(}\DecValTok{1}\FloatTok{-0.54}\NormalTok{)}\OperatorTok{/}\DecValTok{293}\NormalTok{))}

\NormalTok{z_score}
\end{Highlighting}
\end{Shaded}

\begin{verbatim}
## [1] -2.655453
\end{verbatim}

Now that we have a z-score, we can calculate a p-value. Our hypotheses
call for a two-sided test, so:

\begin{Shaded}
\begin{Highlighting}[]
\NormalTok{p_value =}\StringTok{ }\DecValTok{2} \OperatorTok{*}\StringTok{ }\KeywordTok{pnorm}\NormalTok{(}\DataTypeTok{q=}\OperatorTok{-}\FloatTok{2.66}\NormalTok{, }\DataTypeTok{mean=}\DecValTok{0}\NormalTok{, }\DataTypeTok{sd=}\DecValTok{1}\NormalTok{)}

\NormalTok{p_value}
\end{Highlighting}
\end{Shaded}

\begin{verbatim}
## [1] 0.007814065
\end{verbatim}

\hypertarget{computational-method-with-infer-1}{%
\subparagraph{Computational method with
Infer}\label{computational-method-with-infer-1}}

Using infer we will generate the null distribution using the permutation
method. Two appropriate test statistics are, 1) the difference in
proportions, 2) a z-score. First calculating a z-score.

\begin{Shaded}
\begin{Highlighting}[]
\NormalTok{test_stat <-}\StringTok{ }\NormalTok{df_two }\OperatorTok\StringTok{ }
\StringTok{  }\KeywordTok{specify}\NormalTok{(Type }\OperatorTok{~}\StringTok{ }\NormalTok{Region, }\DataTypeTok{success =} \StringTok{"disk"}\NormalTok{) }\OperatorTok\StringTok{ }
\StringTok{  }\KeywordTok{calculate}\NormalTok{(}\DataTypeTok{stat =} \StringTok{"z"}\NormalTok{, }\DataTypeTok{order =} \KeywordTok{c}\NormalTok{(}\StringTok{"east"}\NormalTok{,}\StringTok{"west"}\NormalTok{))}

\NormalTok{test_stat}
\end{Highlighting}
\end{Shaded}

\begin{verbatim}
## # A tibble: 1 x 1
##    stat
##   <dbl>
## 1 -2.60
\end{verbatim}

\begin{Shaded}
\begin{Highlighting}[]
\NormalTok{null_dist <-}\StringTok{ }\NormalTok{df_two }\OperatorTok\StringTok{ }
\StringTok{  }\KeywordTok{specify}\NormalTok{(Type }\OperatorTok{~}\StringTok{ }\NormalTok{Region, }\DataTypeTok{success =} \StringTok{"disk"}\NormalTok{) }\OperatorTok\StringTok{ }
\StringTok{  }\KeywordTok{hypothesise}\NormalTok{(}\DataTypeTok{null=}\StringTok{"independence"}\NormalTok{) }\OperatorTok\StringTok{ }
\StringTok{  }\KeywordTok{generate}\NormalTok{(}\DataTypeTok{reps=}\DecValTok{1000}\NormalTok{,}\DataTypeTok{type=}\StringTok{"permute"}\NormalTok{) }\OperatorTok\StringTok{ }
\StringTok{  }\KeywordTok{calculate}\NormalTok{(}\DataTypeTok{stat =} \StringTok{"z"}\NormalTok{, }\DataTypeTok{order =} \KeywordTok{c}\NormalTok{(}\StringTok{"east"}\NormalTok{,}\StringTok{"west"}\NormalTok{))}

\NormalTok{p_value <-}\StringTok{ }\NormalTok{null_dist }\OperatorTok\StringTok{ }
\StringTok{  }\KeywordTok{get_p_value}\NormalTok{(}\DataTypeTok{obs_stat=}\NormalTok{test_stat, }\DataTypeTok{direction=}\StringTok{"two-sided"}\NormalTok{)}

\NormalTok{p_value}
\end{Highlighting}
\end{Shaded}

\begin{verbatim}
## # A tibble: 1 x 1
##   p_value
##     <dbl>
## 1    0.02
\end{verbatim}

\begin{Shaded}
\begin{Highlighting}[]
\NormalTok{null_dist }\OperatorTok\StringTok{ }\KeywordTok{visualise}\NormalTok{(}\DataTypeTok{obs_stat =}\NormalTok{ test_stat)}
\end{Highlighting}
\end{Shaded}

\begin{verbatim}
## Warning: `visualize()` should no longer be used to plot a p-value.
## Arguments `obs_stat`, `obs_stat_color`, `pvalue_fill`, and `direction` are
## deprecated. Use `shade_p_value()` instead.
\end{verbatim}

\includegraphics{inference_exam_prep_files/figure-latex/unnamed-chunk-9-1.pdf}
Now we will use the difference in proportions as the test statistic.

\begin{Shaded}
\begin{Highlighting}[]
\NormalTok{test_stat <-}\StringTok{ }\NormalTok{df_two }\OperatorTok\StringTok{ }
\StringTok{  }\KeywordTok{specify}\NormalTok{(Type }\OperatorTok{~}\StringTok{ }\NormalTok{Region, }\DataTypeTok{success =} \StringTok{"disk"}\NormalTok{) }\OperatorTok\StringTok{ }
\StringTok{  }\KeywordTok{calculate}\NormalTok{(}\DataTypeTok{stat =} \StringTok{"diff in props"}\NormalTok{, }\DataTypeTok{order =} \KeywordTok{c}\NormalTok{(}\StringTok{"east"}\NormalTok{,}\StringTok{"west"}\NormalTok{))}

\NormalTok{test_stat}
\end{Highlighting}
\end{Shaded}

\begin{verbatim}
## # A tibble: 1 x 1
##      stat
##     <dbl>
## 1 -0.0982
\end{verbatim}

\begin{Shaded}
\begin{Highlighting}[]
\NormalTok{null_dist <-}\StringTok{ }\NormalTok{df_two }\OperatorTok\StringTok{ }
\StringTok{  }\KeywordTok{specify}\NormalTok{(Type }\OperatorTok{~}\StringTok{ }\NormalTok{Region, }\DataTypeTok{success =} \StringTok{"disk"}\NormalTok{) }\OperatorTok\StringTok{ }
\StringTok{  }\KeywordTok{hypothesise}\NormalTok{(}\DataTypeTok{null=}\StringTok{"independence"}\NormalTok{) }\OperatorTok\StringTok{ }
\StringTok{  }\KeywordTok{generate}\NormalTok{(}\DataTypeTok{reps=}\DecValTok{1000}\NormalTok{,}\DataTypeTok{type=}\StringTok{"permute"}\NormalTok{) }\OperatorTok\StringTok{ }
\StringTok{  }\KeywordTok{calculate}\NormalTok{(}\DataTypeTok{stat =} \StringTok{"diff in props"}\NormalTok{, }\DataTypeTok{order =} \KeywordTok{c}\NormalTok{(}\StringTok{"east"}\NormalTok{,}\StringTok{"west"}\NormalTok{))}

\NormalTok{p_value <-}\StringTok{ }\NormalTok{null_dist }\OperatorTok\StringTok{ }
\StringTok{  }\KeywordTok{get_p_value}\NormalTok{(}\DataTypeTok{obs_stat=}\NormalTok{test_stat, }\DataTypeTok{direction=}\StringTok{"two-sided"}\NormalTok{)}

\NormalTok{p_value}
\end{Highlighting}
\end{Shaded}

\begin{verbatim}
## # A tibble: 1 x 1
##   p_value
##     <dbl>
## 1   0.014
\end{verbatim}

\begin{Shaded}
\begin{Highlighting}[]
\NormalTok{null_dist }\OperatorTok\StringTok{ }\KeywordTok{visualise}\NormalTok{(}\DataTypeTok{obs_stat=}\NormalTok{test_stat)}
\end{Highlighting}
\end{Shaded}

\begin{verbatim}
## Warning: `visualize()` should no longer be used to plot a p-value.
## Arguments `obs_stat`, `obs_stat_color`, `pvalue_fill`, and `direction` are
## deprecated. Use `shade_p_value()` instead.
\end{verbatim}

\includegraphics{inference_exam_prep_files/figure-latex/unnamed-chunk-10-1.pdf}

This p-value is quite small, and is strong evidence to reject the null
hypothesis. Therefore, we conclude that the proportion of stars with
disks differs between the ``east'' and ``west'' regions.

\hypertarget{more-than-two-categories}{%
\subsubsection{More than two
Categories}\label{more-than-two-categories}}

\hypertarget{confidence-intervals-1}{%
\subparagraph{Confidence Intervals}\label{confidence-intervals-1}}

Does it make sense to calculate a confidence interval when there is more
than one category?

\hypertarget{hypothesis-testing-1}{%
\subparagraph{Hypothesis testing}\label{hypothesis-testing-1}}

We are still asking the question, ``Is disk proportion related to
Region''. Our null hypothesis is that disk proportion is independent of
region and thus our alternative hypothesis is that disk proportion is
dependent on region.

We will formulate our hypotheses as: \[
\begin{aligned}
H_{0} &: p_{east}  = p_{west} = p_{halo}  \\
H_{A} &: at~least~one~p~is~different
\end{aligned}
\]

Approximationb with Probability Models

We can use the \(\Chi^{2}\) distribution to test this hypothesis. First
lets look at the two way table.

\begin{longtable}[]{@{}lllll@{}}
\toprule
counts & east & west & halo & total\tabularnewline
\midrule
\endhead
disk & 192 & 159 & 507 & 858\tabularnewline
no disk & 240 & 134 & 929 & 1303\tabularnewline
total & 432 & 293 & 1436 & 2161\tabularnewline
\bottomrule
\end{longtable}

We can estimate the number of expected counts in any cell using the
following: \$\$

Expected\_\{i,j\} = \frac{n_{row~i} \times n_{column~j} }{ table~total }

\$\$ Our table of expected counts is then

\begin{longtable}[]{@{}lllll@{}}
\toprule
\begin{minipage}[b]{0.17\columnwidth}\raggedright
counts\strut
\end{minipage} & \begin{minipage}[b]{0.17\columnwidth}\raggedright
east\strut
\end{minipage} & \begin{minipage}[b]{0.17\columnwidth}\raggedright
west\strut
\end{minipage} & \begin{minipage}[b]{0.17\columnwidth}\raggedright
halo\strut
\end{minipage} & \begin{minipage}[b]{0.17\columnwidth}\raggedright
total\strut
\end{minipage}\tabularnewline
\midrule
\endhead
\begin{minipage}[t]{0.17\columnwidth}\raggedright
disk\strut
\end{minipage} & \begin{minipage}[t]{0.17\columnwidth}\raggedright
\(\frac{858 \times 432}{2161}\) = 171.5\strut
\end{minipage} & \begin{minipage}[t]{0.17\columnwidth}\raggedright
\(\frac{858 \times 293}{2161}\) = 116.3\strut
\end{minipage} & \begin{minipage}[t]{0.17\columnwidth}\raggedright
\(\frac{858 \times 1436}{2161}\) = 570.2\strut
\end{minipage} & \begin{minipage}[t]{0.17\columnwidth}\raggedright
858\strut
\end{minipage}\tabularnewline
\begin{minipage}[t]{0.17\columnwidth}\raggedright
no disk\strut
\end{minipage} & \begin{minipage}[t]{0.17\columnwidth}\raggedright
\(\frac{1303 \times 432}{2161}\) = 260.5\strut
\end{minipage} & \begin{minipage}[t]{0.17\columnwidth}\raggedright
\(\frac{1303 \times 293}{2161}\) = 176.7\strut
\end{minipage} & \begin{minipage}[t]{0.17\columnwidth}\raggedright
\(\frac{1303 \times 1436}{2161}\) = 865.9\strut
\end{minipage} & \begin{minipage}[t]{0.17\columnwidth}\raggedright
1303\strut
\end{minipage}\tabularnewline
\begin{minipage}[t]{0.17\columnwidth}\raggedright
total\strut
\end{minipage} & \begin{minipage}[t]{0.17\columnwidth}\raggedright
432\strut
\end{minipage} & \begin{minipage}[t]{0.17\columnwidth}\raggedright
293\strut
\end{minipage} & \begin{minipage}[t]{0.17\columnwidth}\raggedright
1436\strut
\end{minipage} & \begin{minipage}[t]{0.17\columnwidth}\raggedright
2161\strut
\end{minipage}\tabularnewline
\bottomrule
\end{longtable}

To calculate the \(\Chi^{2}\) test statistic, we first calculate a
z-score for each cell.\\
\[
\begin{aligned}
z & = \frac{observed~count - null~count}{Standard~Error} \\
z & = \frac{observed~count - null~count}{\sqrt{null~count}}
\end{aligned}
\] Then each of these terms are squared and added together.

\begin{Shaded}
\begin{Highlighting}[]
\NormalTok{chi_sq_stat <-}\StringTok{ }\NormalTok{( (}\DecValTok{192} \OperatorTok{-}\StringTok{ }\FloatTok{171.5}\NormalTok{)}\OperatorTok{/}\KeywordTok{sqrt}\NormalTok{(}\FloatTok{171.5}\NormalTok{) )}\OperatorTok{^}\DecValTok{2} \OperatorTok{+}\StringTok{ }
\StringTok{          }\NormalTok{( (}\DecValTok{159} \OperatorTok{-}\StringTok{ }\FloatTok{116.3}\NormalTok{)}\OperatorTok{/}\KeywordTok{sqrt}\NormalTok{(}\FloatTok{116.3}\NormalTok{) )}\OperatorTok{^}\DecValTok{2} \OperatorTok{+}\StringTok{ }
\StringTok{          }\NormalTok{( (}\DecValTok{507} \OperatorTok{-}\StringTok{ }\FloatTok{570.2}\NormalTok{)}\OperatorTok{/}\KeywordTok{sqrt}\NormalTok{(}\FloatTok{570.2}\NormalTok{) )}\OperatorTok{^}\DecValTok{2} \OperatorTok{+}\StringTok{ }
\StringTok{          }\NormalTok{( (}\DecValTok{240} \OperatorTok{-}\StringTok{ }\FloatTok{260.5}\NormalTok{)}\OperatorTok{/}\KeywordTok{sqrt}\NormalTok{(}\FloatTok{260.5}\NormalTok{) )}\OperatorTok{^}\DecValTok{2} \OperatorTok{+}\StringTok{ }
\StringTok{          }\NormalTok{( (}\DecValTok{134} \OperatorTok{-}\StringTok{ }\FloatTok{176.7}\NormalTok{)}\OperatorTok{/}\KeywordTok{sqrt}\NormalTok{(}\FloatTok{176.7}\NormalTok{) )}\OperatorTok{^}\DecValTok{2} \OperatorTok{+}\StringTok{ }
\StringTok{          }\NormalTok{( (}\DecValTok{929} \OperatorTok{-}\StringTok{ }\FloatTok{865.9}\NormalTok{)}\OperatorTok{/}\KeywordTok{sqrt}\NormalTok{(}\FloatTok{865.9}\NormalTok{) )}\OperatorTok{^}\DecValTok{2}

\NormalTok{chi_sq_stat}
\end{Highlighting}
\end{Shaded}

\begin{verbatim}
## [1] 41.66293
\end{verbatim}

If \(R\) is the number of categories in the response variable, and \(C\)
is the number of categories in the explanetory variable, then the
degrees of freedom for a \(\Chi^{2}\) test are given by: \[
\begin{aligned}
df & = (R - 1) \times (C-1) \\
df & = (2 - 1) \times (3 -1) \\
df & = 1 \times 2 \\
df & = 2
\end{aligned}
\]

We can now calculate the p-value

\begin{Shaded}
\begin{Highlighting}[]
\NormalTok{p_value <-}\StringTok{ }\DecValTok{1} \OperatorTok{-}\StringTok{ }\KeywordTok{pchisq}\NormalTok{(}\DataTypeTok{q=}\NormalTok{chi_sq_stat,}\DataTypeTok{df=}\DecValTok{2}\NormalTok{)}

\NormalTok{p_value}
\end{Highlighting}
\end{Shaded}

\begin{verbatim}
## [1] 8.974491e-10
\end{verbatim}

We can use the R function \texttt{chisq.test()} to perform a hypothesis
test assuming the \(\Chi^{2}\) probability model.

\begin{Shaded}
\begin{Highlighting}[]
\NormalTok{df_three <-}\StringTok{ }\NormalTok{df }\OperatorTok\StringTok{ }
\StringTok{  }\KeywordTok{filter}\NormalTok{(Type }\OperatorTok\StringTok{ }\KeywordTok{c}\NormalTok{(}\StringTok{"disk"}\NormalTok{,}\StringTok{"no disk"}\NormalTok{)) }\OperatorTok\StringTok{ }
\StringTok{  }\KeywordTok{select}\NormalTok{(Type,Region) }\OperatorTok\StringTok{ }
\StringTok{  }\KeywordTok{mutate}\NormalTok{(}\DataTypeTok{Type =} \KeywordTok{fct_drop}\NormalTok{(Type))}

\KeywordTok{chisq.test}\NormalTok{(}\KeywordTok{table}\NormalTok{(df_three}\OperatorTok{$}\NormalTok{Type, df_three}\OperatorTok{$}\NormalTok{Region))}
\end{Highlighting}
\end{Shaded}

\begin{verbatim}
## 
##  Pearson's Chi-squared test
## 
## data:  table(df_three$Type, df_three$Region)
## X-squared = 41.609, df = 2, p-value = 9.22e-10
\end{verbatim}

\hypertarget{computational-method-with-infer-2}{%
\subparagraph{Computational method with
Infer}\label{computational-method-with-infer-2}}

\textbf{Note:} \texttt{fct\_drop()} is used here to remove the empty
``envelope'' category.

\begin{Shaded}
\begin{Highlighting}[]
\NormalTok{df }\OperatorTok\StringTok{ }
\StringTok{  }\KeywordTok{filter}\NormalTok{(Type }\OperatorTok\StringTok{ }\KeywordTok{c}\NormalTok{(}\StringTok{"disk"}\NormalTok{,}\StringTok{"no disk"}\NormalTok{)) }\OperatorTok\StringTok{ }
\StringTok{  }\KeywordTok{select}\NormalTok{(Type,Region) }\OperatorTok\StringTok{ }
\StringTok{  }\KeywordTok{group_by}\NormalTok{(Region,Type) }\OperatorTok\StringTok{ }
\StringTok{  }\KeywordTok{summarise}\NormalTok{(}\DataTypeTok{count=}\KeywordTok{n}\NormalTok{())}
\end{Highlighting}
\end{Shaded}

\begin{verbatim}
## # A tibble: 6 x 3
## # Groups:   Region [3]
##   Region Type    count
##   <chr>  <fct>   <int>
## 1 east   disk      192
## 2 east   no disk   240
## 3 halo   disk      507
## 4 halo   no disk   929
## 5 west   disk      159
## 6 west   no disk   134
\end{verbatim}

\begin{Shaded}
\begin{Highlighting}[]
\NormalTok{df_three <-}\StringTok{ }\NormalTok{df }\OperatorTok\StringTok{ }
\StringTok{  }\KeywordTok{filter}\NormalTok{(Type }\OperatorTok\StringTok{ }\KeywordTok{c}\NormalTok{(}\StringTok{"disk"}\NormalTok{,}\StringTok{"no disk"}\NormalTok{)) }\OperatorTok\StringTok{ }
\StringTok{  }\KeywordTok{select}\NormalTok{(Type,Region) }\OperatorTok\StringTok{ }
\StringTok{  }\KeywordTok{mutate}\NormalTok{(}\DataTypeTok{Type =} \KeywordTok{fct_drop}\NormalTok{(Type))}

\NormalTok{test_stat <-}\StringTok{ }\NormalTok{df_three }\OperatorTok\StringTok{ }
\StringTok{  }\KeywordTok{specify}\NormalTok{(Type }\OperatorTok{~}\StringTok{ }\NormalTok{Region) }\OperatorTok
\StringTok{  }\CommentTok{#hypothesize(null = "independence") %>%}
\StringTok{  }\CommentTok{#generate(reps = 1000, type = "permute") %>%}
\StringTok{  }\KeywordTok{calculate}\NormalTok{(}\DataTypeTok{stat =} \StringTok{"Chisq"}\NormalTok{)}


\NormalTok{test_stat}
\end{Highlighting}
\end{Shaded}

\begin{verbatim}
## # A tibble: 1 x 1
##    stat
##   <dbl>
## 1  41.6
\end{verbatim}

\begin{Shaded}
\begin{Highlighting}[]
\NormalTok{null_dist <-}\StringTok{ }\NormalTok{df_three }\OperatorTok\StringTok{ }
\StringTok{  }\KeywordTok{specify}\NormalTok{(Type }\OperatorTok{~}\StringTok{ }\NormalTok{Region) }\OperatorTok
\StringTok{  }\KeywordTok{hypothesize}\NormalTok{(}\DataTypeTok{null =} \StringTok{"independence"}\NormalTok{) }\OperatorTok
\StringTok{  }\KeywordTok{generate}\NormalTok{(}\DataTypeTok{reps =} \DecValTok{1000}\NormalTok{, }\DataTypeTok{type =} \StringTok{"permute"}\NormalTok{) }\OperatorTok
\StringTok{  }\KeywordTok{calculate}\NormalTok{(}\DataTypeTok{stat =} \StringTok{"Chisq"}\NormalTok{)}

\NormalTok{null_dist}
\end{Highlighting}
\end{Shaded}

\begin{verbatim}
## # A tibble: 1,000 x 2
##    replicate  stat
##        <int> <dbl>
##  1         1 0.257
##  2         2 9.17 
##  3         3 0.228
##  4         4 4.06 
##  5         5 0.390
##  6         6 0.824
##  7         7 9.27 
##  8         8 0.691
##  9         9 5.19 
## 10        10 1.67 
## # ... with 990 more rows
\end{verbatim}

\begin{Shaded}
\begin{Highlighting}[]
\NormalTok{p_value <-}\StringTok{ }\NormalTok{null_dist }\OperatorTok\StringTok{ }
\StringTok{  }\KeywordTok{get_p_value}\NormalTok{(}\DataTypeTok{obs_stat =}\NormalTok{ test_stat, }\DataTypeTok{direction=}\StringTok{"greater"}\NormalTok{)}
\end{Highlighting}
\end{Shaded}

\begin{verbatim}
## Warning: Please be cautious in reporting a p-value of 0. This result is
## an approximation based on the number of `reps` chosen in the `generate()`
## step. See `?get_p_value()` for more information.
\end{verbatim}

\begin{Shaded}
\begin{Highlighting}[]
\NormalTok{p_value}
\end{Highlighting}
\end{Shaded}

\begin{verbatim}
## # A tibble: 1 x 1
##   p_value
##     <dbl>
## 1       0
\end{verbatim}

\begin{Shaded}
\begin{Highlighting}[]
\NormalTok{null_dist }\OperatorTok\StringTok{ }\KeywordTok{visualise}\NormalTok{(}\DataTypeTok{obs_stat=}\NormalTok{test_stat)}
\end{Highlighting}
\end{Shaded}

\begin{verbatim}
## Warning: `visualize()` should no longer be used to plot a p-value.
## Arguments `obs_stat`, `obs_stat_color`, `pvalue_fill`, and `direction` are
## deprecated. Use `shade_p_value()` instead.
\end{verbatim}

\includegraphics{inference_exam_prep_files/figure-latex/unnamed-chunk-14-1.pdf}

As we see, regardless of method used, the test statistic is far into the
wings of the null distribution. Therefore, we consider that to be stron
evidence to reject the null hypothesis that disk proportion and region
are independent. We conclude then, that disk proprtion depends on which
region a star is in.

\hypertarget{problem-2}{%
\subsection{Problem 2}\label{problem-2}}

We are now interested in exploring whether there are difference in the
type of light emitted from each \texttt{Type} of star. We will do this
by creating a new \emph{color index} composed of the measurements made
in certain Bands. To do this create a new variable called \texttt{Color}
made of the difference \texttt{Band8}-\texttt{Band9}. You will want to
remove \texttt{NA}'s from this variable.

\begin{center}\rule{0.5\linewidth}{\linethickness}\end{center}

\hypertarget{exploratory-data-analysis-1}{%
\subsubsection{Exploratory Data
Analysis}\label{exploratory-data-analysis-1}}

\begin{Shaded}
\begin{Highlighting}[]
\NormalTok{df_colors <-}\StringTok{ }\NormalTok{df }\OperatorTok\StringTok{ }
\StringTok{  }\KeywordTok{mutate}\NormalTok{(}\DataTypeTok{Color =}\NormalTok{ Band8 }\OperatorTok{-}\StringTok{ }\NormalTok{Band9) }\OperatorTok\StringTok{ }
\StringTok{  }\CommentTok{#mutate(Color = Band1 - Band2) %>% }
\StringTok{  }\KeywordTok{drop_na}\NormalTok{(Color)}


\CommentTok{# Boxplot of Index as function of Type (Disk or No Disk)}
\NormalTok{df_colors }\OperatorTok\StringTok{ }
\StringTok{  }\CommentTok{#drop_na(Band3,Band4,Band5,Band6,Band7,Band8,Band9) %>% }
\StringTok{  }\CommentTok{#mutate(Color1 = Band3-Band4,}
\StringTok{  }\CommentTok{#       Color2 = Band6 - Band7, }
\StringTok{  }\CommentTok{#       Color3 = Band8 - Band9) %>% }
\StringTok{  }\KeywordTok{ggplot}\NormalTok{(}\KeywordTok{aes}\NormalTok{(}\DataTypeTok{x=}\NormalTok{Type, }\DataTypeTok{y=}\NormalTok{Color)) }\OperatorTok{+}
\StringTok{    }\KeywordTok{geom_boxplot}\NormalTok{()}
\end{Highlighting}
\end{Shaded}

\includegraphics{inference_exam_prep_files/figure-latex/unnamed-chunk-15-1.pdf}

\begin{Shaded}
\begin{Highlighting}[]
\NormalTok{df_color_table <-}\StringTok{ }\NormalTok{df_colors }\OperatorTok\StringTok{ }
\StringTok{  }\KeywordTok{group_by}\NormalTok{(Type) }\OperatorTok\StringTok{ }
\StringTok{  }\KeywordTok{summarize}\NormalTok{(}\DataTypeTok{mean =} \KeywordTok{mean}\NormalTok{(Color),}\DataTypeTok{sd=}\KeywordTok{sd}\NormalTok{(Color),}\DataTypeTok{count=}\KeywordTok{n}\NormalTok{())}
  
\NormalTok{df_color_table}
\end{Highlighting}
\end{Shaded}

\begin{verbatim}
## # A tibble: 3 x 4
##   Type       mean    sd count
##   <fct>     <dbl> <dbl> <int>
## 1 envelope 1.14   0.152     5
## 2 disk     0.673  0.254   833
## 3 no disk  0.0266 0.229  1206
\end{verbatim}

\hypertarget{two-categories-1}{%
\subsubsection{Two Categories}\label{two-categories-1}}

\hypertarget{confidence-intervals-2}{%
\paragraph{Confidence intervals}\label{confidence-intervals-2}}

First, we just consider the regions ``disk'' and ``no disk'', and for
eash type estimate the mean value of the color index and determine
\(95\%\) confidence intervals.

\begin{center}\rule{0.5\linewidth}{\linethickness}\end{center}

\hypertarget{approximation-with-probability-models-1}{%
\subparagraph{Approximation with probability
models}\label{approximation-with-probability-models-1}}

For a single mean point estimate: \[
\begin{aligned}
\bar{x} & \pm MOE \\
\bar{x} & \pm t_{df}^{*} \times \frac{s}{\sqrt{n}}
\end{aligned}
\] The ``disk'' stars have a mean \texttt{Color} of 0.6732893 with n =
833 observations, a sample standard deviation, s = 0.2537412 and df =
n-1 = 832 degrees of freedom. For a sample of this size
\(t_{df}^{*} \simeq z^{*}\). \[
\begin{aligned}
\bar{x_{disk}} & \pm z^{*} \times \frac{s}{\sqrt{n}} \\
0.6732893 & \pm 1.96 \times \frac{0.2537412}{\sqrt{833}} \\
0.6732893 & \pm 0.02 
\end{aligned}
\]

R can be used as a calculator.

\begin{Shaded}
\begin{Highlighting}[]
\FloatTok{1.96} \OperatorTok{*}\StringTok{ }\NormalTok{df_color_table[df_color_table}\OperatorTok{$}\NormalTok{Type }\OperatorTok{==}\StringTok{ "disk"}\NormalTok{,}\StringTok{"sd"}\NormalTok{] }\OperatorTok{/}\StringTok{ }\KeywordTok{sqrt}\NormalTok{(df_color_table[df_color_table}\OperatorTok{$}\NormalTok{Type }\OperatorTok{==}\StringTok{ "disk"}\NormalTok{,}\StringTok{"count"}\NormalTok{])}
\end{Highlighting}
\end{Shaded}

\begin{verbatim}
##           sd
## 1 0.01723156
\end{verbatim}

\begin{Shaded}
\begin{Highlighting}[]
\CommentTok{#1.96 *  0.25 / sqrt(833) # If Color = Band8 - Band9}
\end{Highlighting}
\end{Shaded}

The ``no disk'' stars have a mean \texttt{Color} of 0.0265755 with n =
1206 observations, a sample standard deviation, s = 0.2294564 and df =
n-1 = 1205 degrees of freedom. Again, for a sample of this size
\(t_{df}^{*} \simeq z^{*}\). \[
\begin{aligned}
\bar{x_{disk}} & \pm z^{*} \times \frac{s}{\sqrt{n}} \\
0.0265755 & \pm 1.96 \times \frac{0.2294564}{\sqrt{1206}} \\
0.0265755 & \pm 0.01
\end{aligned}
\]

\begin{Shaded}
\begin{Highlighting}[]
\FloatTok{1.96} \OperatorTok{*}\StringTok{ }\NormalTok{df_color_table[df_color_table}\OperatorTok{$}\NormalTok{Type }\OperatorTok{==}\StringTok{ "no disk"}\NormalTok{,}\StringTok{"sd"}\NormalTok{] }\OperatorTok{/}\StringTok{ }\KeywordTok{sqrt}\NormalTok{(df_color_table[df_color_table}\OperatorTok{$}\NormalTok{Type }\OperatorTok{==}\StringTok{ "no disk"}\NormalTok{,}\StringTok{"count"}\NormalTok{])}
\end{Highlighting}
\end{Shaded}

\begin{verbatim}
##           sd
## 1 0.01295038
\end{verbatim}

\begin{Shaded}
\begin{Highlighting}[]
\FloatTok{1.96} \OperatorTok{*}\StringTok{  }\FloatTok{0.23} \OperatorTok{/}\StringTok{ }\KeywordTok{sqrt}\NormalTok{(}\DecValTok{1206}\NormalTok{)}
\end{Highlighting}
\end{Shaded}

\begin{verbatim}
## [1] 0.01298106
\end{verbatim}

\hypertarget{computational-method-with-infer-3}{%
\subparagraph{Computational method with
Infer}\label{computational-method-with-infer-3}}

We will use \texttt{infer} to construct bootstrap estimations of the
sampling distribution.

\begin{Shaded}
\begin{Highlighting}[]
\CommentTok{# First }
 
\NormalTok{df_two <-}\StringTok{ }\NormalTok{df_colors }\OperatorTok\StringTok{ }
\StringTok{  }\KeywordTok{filter}\NormalTok{(Type }\OperatorTok\StringTok{ }\KeywordTok{c}\NormalTok{(}\StringTok{"disk"}\NormalTok{,}\StringTok{"no disk"}\NormalTok{)) }\OperatorTok\StringTok{ }
\StringTok{  }\KeywordTok{mutate}\NormalTok{(}\DataTypeTok{Type =} \KeywordTok{fct_drop}\NormalTok{(Type))}

\NormalTok{df_two}
\end{Highlighting}
\end{Shaded}

\begin{verbatim}
## # A tibble: 2,039 x 28
##       RA   Dec Type   Band1 Band2 Band3 Band4 Band5 Band6 Band7 Band8 Band9
##    <dbl> <dbl> <fct>  <dbl> <dbl> <dbl> <dbl> <dbl> <dbl> <dbl> <dbl> <dbl>
##  1  344.  62.6 no di~  15.1  13.3  12.1  11.6  11.4  11.2  11.2  11.2  11.2
##  2  344.  62.8 no di~  17.6  15.2  13.8  13.2  12.9  12.7  12.7  12.6  12.7
##  3  344.  62.7 no di~  15.9  13.8  12.0  11.2  10.9  10.8  10.8  10.7  10.7
##  4  344.  62.6 no di~  20.2  17.1  15.0  14.1  13.7  13.4  13.4  13.2  13.3
##  5  344.  62.5 no di~  18.8  16.1  14.1  13.2  12.9  12.7  12.6  12.6  12.5
##  6  344.  62.4 no di~  19.6  16.4  14.2  13.3  13.0  12.6  12.6  12.4  11.8
##  7  344.  62.6 no di~  22.2  18.1  15.8  14.8  14.3  14.0  13.9  14.0  13.9
##  8  344.  62.5 no di~  19.4  15.9  13.2  12.0  11.6  11.2  11.2  11.1  11.2
##  9  344.  62.4 no di~  21.2  17.5  14.7  13.5  13.1  12.7  12.7  12.7  12.6
## 10  344.  62.5 no di~  20.1  16.9  14.6  13.5  13.2  13.0  12.9  12.9  13.0
## # ... with 2,029 more rows, and 16 more variables: Band10 <dbl>,
## #   Dust <dbl>, Xray <chr>, parallax <dbl>, parallax_error <dbl>,
## #   pmra <dbl>, pmra_error <dbl>, pmdec <dbl>, pmdec_error <dbl>,
## #   teff_val <dbl>, a_g_val <dbl>, radius_val <dbl>, lum_val <dbl>,
## #   Distance <dbl>, Region <chr>, Color <dbl>
\end{verbatim}

\begin{Shaded}
\begin{Highlighting}[]
\CommentTok{# disk}
\NormalTok{mean_disk <-}\StringTok{ }\NormalTok{df_two }\OperatorTok\StringTok{ }
\StringTok{  }\KeywordTok{filter}\NormalTok{(Type }\OperatorTok{==}\StringTok{ "disk"}\NormalTok{) }\OperatorTok\StringTok{ }
\StringTok{  }\KeywordTok{specify}\NormalTok{(}\DataTypeTok{response =}\NormalTok{ Color) }\OperatorTok\StringTok{ }
\StringTok{  }\KeywordTok{calculate}\NormalTok{(}\DataTypeTok{stat =} \StringTok{"mean"}\NormalTok{)}

\NormalTok{boot_disk <-}\StringTok{ }\NormalTok{df_two }\OperatorTok\StringTok{ }
\StringTok{  }\KeywordTok{filter}\NormalTok{(Type }\OperatorTok{==}\StringTok{ "disk"}\NormalTok{) }\OperatorTok\StringTok{ }
\StringTok{  }\KeywordTok{specify}\NormalTok{(}\DataTypeTok{response =}\NormalTok{ Color) }\OperatorTok\StringTok{ }
\StringTok{  }\KeywordTok{generate}\NormalTok{(}\DataTypeTok{reps=}\DecValTok{1000}\NormalTok{,}\DataTypeTok{type=}\StringTok{"bootstrap"}\NormalTok{) }\OperatorTok\StringTok{ }
\StringTok{  }\KeywordTok{calculate}\NormalTok{(}\DataTypeTok{stat =} \StringTok{"mean"}\NormalTok{) }

\NormalTok{ci_disk <-}\StringTok{ }\NormalTok{boot_disk }\OperatorTok\StringTok{ }
\StringTok{  }\KeywordTok{get_confidence_interval}\NormalTok{(}\DataTypeTok{type =} \StringTok{"se"}\NormalTok{, }\DataTypeTok{point_estimate =}\NormalTok{ mean_disk)}
\end{Highlighting}
\end{Shaded}

\begin{verbatim}
## Using `level = 0.95` to compute confidence interval.
\end{verbatim}

\begin{Shaded}
\begin{Highlighting}[]
\NormalTok{mean_disk}
\end{Highlighting}
\end{Shaded}

\begin{verbatim}
## # A tibble: 1 x 1
##    stat
##   <dbl>
## 1 0.673
\end{verbatim}

\begin{Shaded}
\begin{Highlighting}[]
\NormalTok{ci_disk}
\end{Highlighting}
\end{Shaded}

\begin{verbatim}
## # A tibble: 1 x 2
##   lower_ci upper_ci
##      <dbl>    <dbl>
## 1    0.656    0.691
\end{verbatim}

\begin{Shaded}
\begin{Highlighting}[]
\CommentTok{# no disk}
\NormalTok{mean_nodisk <-}\StringTok{ }\NormalTok{df_two }\OperatorTok\StringTok{ }
\StringTok{  }\KeywordTok{filter}\NormalTok{(Type }\OperatorTok{==}\StringTok{ "no disk"}\NormalTok{) }\OperatorTok\StringTok{ }
\StringTok{  }\KeywordTok{specify}\NormalTok{(}\DataTypeTok{response =}\NormalTok{ Color) }\OperatorTok\StringTok{ }
\StringTok{  }\KeywordTok{calculate}\NormalTok{(}\DataTypeTok{stat =} \StringTok{"mean"}\NormalTok{)}

\NormalTok{boot_nodisk <-}\StringTok{ }\NormalTok{df_two }\OperatorTok\StringTok{ }
\StringTok{  }\KeywordTok{filter}\NormalTok{(Type }\OperatorTok{==}\StringTok{ "no disk"}\NormalTok{) }\OperatorTok\StringTok{ }
\StringTok{  }\KeywordTok{specify}\NormalTok{(}\DataTypeTok{response =}\NormalTok{ Color) }\OperatorTok\StringTok{ }
\StringTok{  }\KeywordTok{generate}\NormalTok{(}\DataTypeTok{reps=}\DecValTok{1000}\NormalTok{,}\DataTypeTok{type=}\StringTok{"bootstrap"}\NormalTok{) }\OperatorTok\StringTok{ }
\StringTok{  }\KeywordTok{calculate}\NormalTok{(}\DataTypeTok{stat =} \StringTok{"mean"}\NormalTok{) }

\NormalTok{ci_nodisk <-}\StringTok{ }\NormalTok{boot_nodisk }\OperatorTok\StringTok{ }
\StringTok{  }\KeywordTok{get_confidence_interval}\NormalTok{(}\DataTypeTok{type =} \StringTok{"se"}\NormalTok{, }\DataTypeTok{point_estimate =}\NormalTok{ mean_nodisk)}
\end{Highlighting}
\end{Shaded}

\begin{verbatim}
## Using `level = 0.95` to compute confidence interval.
\end{verbatim}

\begin{Shaded}
\begin{Highlighting}[]
\NormalTok{mean_nodisk}
\end{Highlighting}
\end{Shaded}

\begin{verbatim}
## # A tibble: 1 x 1
##     stat
##    <dbl>
## 1 0.0266
\end{verbatim}

\begin{Shaded}
\begin{Highlighting}[]
\NormalTok{ci_nodisk}
\end{Highlighting}
\end{Shaded}

\begin{verbatim}
## # A tibble: 1 x 2
##   lower_ci upper_ci
##      <dbl>    <dbl>
## 1   0.0129   0.0402
\end{verbatim}

\hypertarget{hypothesis-testing-2}{%
\subparagraph{Hypothesis testing}\label{hypothesis-testing-2}}

Now answer the question, ``Does \texttt{Color} depend on
\texttt{Type}?'' Formulate a null and alterntive hypothesis, calculate a
test statistic and a p-value and then make a conclusion about your
hypotheis.

We will formulate our hypotheses as: \$\$
\textbackslash{}begin\{aligned\} H\_\{0\} \&: \mu\emph{\{disk\} -
\mu}\{no\textasciitilde{}disk\} = 0 \textbackslash{} H\_\{A\} \&:
\mu\emph{\{disk\} - \mu}\{no\textasciitilde{}disk\} \ne 0

\textbackslash{}end\{aligned\} \$\$

\hypertarget{approximation-with-probability-models-2}{%
\subparagraph{Approximation with probability
models}\label{approximation-with-probability-models-2}}

We will first approximate the null distribution using a t-distribution.

\[
\begin{aligned}
t & = \frac{ \bar{x_{disk}}-\bar{x_{no~disk}} - 0 }{\sqrt{ \frac{s_{disk}^{2}}{n_{disk}} +
  \frac{s_{no~disk}^{2}}{n_{no~disk}}  }} \\
t & = \frac{ \bar{0.6732893}-\bar{0.0265755} - 0 }{\sqrt{ \frac{0.2294564^{2}}{833} +
  \frac{0.2294564^{2}}{1206}  }} \\
0.03 & \pm 0.01
\end{aligned}
\]

Using R as a calculator:

\begin{Shaded}
\begin{Highlighting}[]
\NormalTok{t_stat <-}\StringTok{ }\NormalTok{(df_color_table[df_color_table}\OperatorTok{$}\NormalTok{Type }\OperatorTok{==}\StringTok{ "disk"}\NormalTok{,}\StringTok{"mean"}\NormalTok{] }\OperatorTok{-}\StringTok{ }\NormalTok{df_color_table[df_color_table}\OperatorTok{$}\NormalTok{Type }\OperatorTok{==}\StringTok{ "no disk"}\NormalTok{,}\StringTok{"mean"}\NormalTok{]) }\OperatorTok{/}\StringTok{ }\KeywordTok{sqrt}\NormalTok{ ( (df_color_table[df_color_table}\OperatorTok{$}\NormalTok{Type }\OperatorTok{==}\StringTok{ "disk"}\NormalTok{,}\StringTok{"sd"}\NormalTok{]}\OperatorTok{^}\DecValTok{2}\NormalTok{)}\OperatorTok{/}\NormalTok{df_color_table[df_color_table}\OperatorTok{$}\NormalTok{Type }\OperatorTok{==}\StringTok{ "disk"}\NormalTok{,}\StringTok{"count"}\NormalTok{] }\OperatorTok{+}\StringTok{ }\NormalTok{(df_color_table[df_color_table}\OperatorTok{$}\NormalTok{Type }\OperatorTok{==}\StringTok{ "no disk"}\NormalTok{,}\StringTok{"sd"}\NormalTok{]}\OperatorTok{^}\DecValTok{2}\NormalTok{)}\OperatorTok{/}\NormalTok{df_color_table[df_color_table}\OperatorTok{$}\NormalTok{Type }\OperatorTok{==}\StringTok{ "no disk"}\NormalTok{,}\StringTok{"count"}\NormalTok{] ) }

\NormalTok{df <-}\StringTok{ }\KeywordTok{min}\NormalTok{(df_color_table[df_color_table}\OperatorTok{$}\NormalTok{Type }\OperatorTok{==}\StringTok{ "no disk"}\NormalTok{,}\StringTok{"count"}\NormalTok{],df_color_table[df_color_table}\OperatorTok{$}\NormalTok{Type }\OperatorTok{==}\StringTok{ "disk"}\NormalTok{,}\StringTok{"count"}\NormalTok{]) }\OperatorTok{-}\StringTok{ }\DecValTok{1}

\CommentTok{#df}
\NormalTok{t_stat <-}\StringTok{ }\NormalTok{t_stat }\OperatorTok\StringTok{ }\KeywordTok{pull}\NormalTok{()}

\NormalTok{t_stat}
\end{Highlighting}
\end{Shaded}

\begin{verbatim}
## [1] 58.80447
\end{verbatim}

And \texttt{pt()} can be used to calculate th p-value:

\begin{Shaded}
\begin{Highlighting}[]
\NormalTok{p_value <-}\StringTok{ }\DecValTok{2}\OperatorTok{*}\NormalTok{(}\DecValTok{1} \OperatorTok{-}\StringTok{ }\KeywordTok{pt}\NormalTok{(}\DataTypeTok{q=}\NormalTok{t_stat,df))}

\NormalTok{p_value}
\end{Highlighting}
\end{Shaded}

\begin{verbatim}
## [1] 0
\end{verbatim}

Another approach is to use the R function \texttt{t.test()}.

\begin{Shaded}
\begin{Highlighting}[]
\NormalTok{df_two <-}\StringTok{ }\NormalTok{df_colors }\OperatorTok\StringTok{ }
\StringTok{  }\KeywordTok{filter}\NormalTok{(Type }\OperatorTok\StringTok{ }\KeywordTok{c}\NormalTok{(}\StringTok{"disk"}\NormalTok{,}\StringTok{"no disk"}\NormalTok{)) }\OperatorTok\StringTok{ }
\StringTok{  }\KeywordTok{mutate}\NormalTok{(}\DataTypeTok{Type =} \KeywordTok{fct_drop}\NormalTok{(Type))}

\KeywordTok{t.test}\NormalTok{(Color }\OperatorTok{~}\StringTok{ }\NormalTok{Type,}\DataTypeTok{data=}\NormalTok{df_two)}
\end{Highlighting}
\end{Shaded}

\begin{verbatim}
## 
##  Welch Two Sample t-test
## 
## data:  Color by Type
## t = 58.804, df = 1669.5, p-value < 2.2e-16
## alternative hypothesis: true difference in means is not equal to 0
## 95 percent confidence interval:
##  0.6251431 0.6682846
## sample estimates:
##    mean in group disk mean in group no disk 
##            0.67328932            0.02657546
\end{verbatim}

\hypertarget{computational-method-with-infer-4}{%
\subparagraph{Computational method with
Infer}\label{computational-method-with-infer-4}}

\begin{Shaded}
\begin{Highlighting}[]
\NormalTok{test_stat <-}\StringTok{ }\NormalTok{df_two }\OperatorTok\StringTok{ }
\StringTok{  }\KeywordTok{specify}\NormalTok{(Color }\OperatorTok{~}\StringTok{ }\NormalTok{Type) }\OperatorTok\StringTok{ }
\StringTok{  }\KeywordTok{calculate}\NormalTok{(}\DataTypeTok{stat =} \StringTok{"t"}\NormalTok{,}\DataTypeTok{order =} \KeywordTok{c}\NormalTok{(}\StringTok{"disk"}\NormalTok{, }\StringTok{"no disk"}\NormalTok{))}

\NormalTok{test_stat}
\end{Highlighting}
\end{Shaded}

\begin{verbatim}
## # A tibble: 1 x 1
##    stat
##   <dbl>
## 1  58.8
\end{verbatim}

\begin{Shaded}
\begin{Highlighting}[]
\NormalTok{null_dist <-}\StringTok{ }\NormalTok{df_two }\OperatorTok\StringTok{ }
\StringTok{  }\KeywordTok{specify}\NormalTok{(Color }\OperatorTok{~}\StringTok{ }\NormalTok{Type) }\OperatorTok\StringTok{ }
\StringTok{  }\KeywordTok{hypothesise}\NormalTok{(}\DataTypeTok{null=}\StringTok{"independence"}\NormalTok{) }\OperatorTok\StringTok{ }
\StringTok{  }\KeywordTok{generate}\NormalTok{(}\DataTypeTok{reps=}\DecValTok{1000}\NormalTok{,}\DataTypeTok{type=}\StringTok{"permute"}\NormalTok{) }\OperatorTok\StringTok{ }
\StringTok{  }\KeywordTok{calculate}\NormalTok{(}\DataTypeTok{stat =} \StringTok{"t"}\NormalTok{,}\DataTypeTok{order =} \KeywordTok{c}\NormalTok{(}\StringTok{"disk"}\NormalTok{, }\StringTok{"no disk"}\NormalTok{))}

\NormalTok{p_value <-}\StringTok{ }\NormalTok{null_dist }\OperatorTok\StringTok{ }
\StringTok{  }\KeywordTok{get_p_value}\NormalTok{(}\DataTypeTok{obs_stat=}\NormalTok{test_stat, }\DataTypeTok{direction=}\StringTok{"two-sided"}\NormalTok{)}
\end{Highlighting}
\end{Shaded}

\begin{verbatim}
## Warning: Please be cautious in reporting a p-value of 0. This result is
## an approximation based on the number of `reps` chosen in the `generate()`
## step. See `?get_p_value()` for more information.
\end{verbatim}

\begin{Shaded}
\begin{Highlighting}[]
\NormalTok{p_value}
\end{Highlighting}
\end{Shaded}

\begin{verbatim}
## # A tibble: 1 x 1
##   p_value
##     <dbl>
## 1       0
\end{verbatim}

\hypertarget{more-than-two-categories-1}{%
\subsubsection{More than two
Categories}\label{more-than-two-categories-1}}

We now consider all three types: ``envelope'', ``disk'' and ``no disk''.

We will formulate our hypotheses as: \[
\begin{aligned}
H_{0} &: \mu_{disk}  = \mu_{no~disk} = \mu_{envelope}  \\
H_{A} &: at~least~one~\mu~is~different
\end{aligned}
\]

Approximationb with Probability Models

\hypertarget{computational-method-with-infer-5}{%
\subparagraph{Computational method with
Infer}\label{computational-method-with-infer-5}}

\begin{Shaded}
\begin{Highlighting}[]
\NormalTok{f_stat <-}\StringTok{ }\NormalTok{df_colors }\OperatorTok\StringTok{ }
\StringTok{  }\KeywordTok{specify}\NormalTok{(Color }\OperatorTok{~}\StringTok{ }\NormalTok{Type) }\OperatorTok\StringTok{ }
\StringTok{  }\KeywordTok{calculate}\NormalTok{(}\DataTypeTok{stat =} \StringTok{"F"}\NormalTok{)}

\NormalTok{f_stat}
\end{Highlighting}
\end{Shaded}

\begin{verbatim}
## # A tibble: 1 x 1
##    stat
##   <dbl>
## 1 1827.
\end{verbatim}

\begin{Shaded}
\begin{Highlighting}[]
\NormalTok{null_dist <-}\StringTok{ }\NormalTok{df_colors }\OperatorTok\StringTok{ }
\StringTok{  }\KeywordTok{specify}\NormalTok{(Color }\OperatorTok{~}\StringTok{ }\NormalTok{Type) }\OperatorTok\StringTok{ }
\StringTok{  }\KeywordTok{hypothesise}\NormalTok{(}\DataTypeTok{null=}\StringTok{"independence"}\NormalTok{) }\OperatorTok\StringTok{ }
\StringTok{  }\KeywordTok{generate}\NormalTok{(}\DataTypeTok{reps=}\DecValTok{1000}\NormalTok{,}\DataTypeTok{type=}\StringTok{"permute"}\NormalTok{) }\OperatorTok\StringTok{ }
\StringTok{  }\KeywordTok{calculate}\NormalTok{(}\DataTypeTok{stat =} \StringTok{"F"}\NormalTok{)}


\NormalTok{p_value <-}\StringTok{ }\NormalTok{null_dist }\OperatorTok\StringTok{ }
\StringTok{  }\KeywordTok{get_p_value}\NormalTok{(}\DataTypeTok{obs_stat=}\NormalTok{f_stat,}\DataTypeTok{direction=}\StringTok{"greater"}\NormalTok{)}
\end{Highlighting}
\end{Shaded}

\begin{verbatim}
## Warning: Please be cautious in reporting a p-value of 0. This result is
## an approximation based on the number of `reps` chosen in the `generate()`
## step. See `?get_p_value()` for more information.
\end{verbatim}

\begin{Shaded}
\begin{Highlighting}[]
\NormalTok{p_value}
\end{Highlighting}
\end{Shaded}

\begin{verbatim}
## # A tibble: 1 x 1
##   p_value
##     <dbl>
## 1       0
\end{verbatim}

Check conditions


\end{document}
